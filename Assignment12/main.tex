%%%%%%%%%%%%%%%%%%%%%%%%%%%%%%%%%%%%%%%%%%%%%%%%%%%%%%%%%%%%%%%
%
% Welcome to Overleaf --- just edit your LaTeX on the left,
% and we'll compile it for you on the right. If you open the
% 'Share' menu, you can invite other users to edit at the same
% time. See www.overleaf.com/learn for more info. Enjoy!
%
%%%%%%%%%%%%%%%%%%%%%%%%%%%%%%%%%%%%%%%%%%%%%%%%%%%%%%%%%%%%%%%

% Inbuilt themes in beamer
\documentclass{beamer}

%packages:
% \usepackage{tfrupee}
% \usepackage{amsmath}
% \usepackage{amssymb}
% \usepackage{gensymb}
% \usepackage{txfonts}

% \def\inputGnumericTable{}

% \usepackage[latin1]{inputenc}                                 
% \usepackage{color}                                            
% \usepackage{array}                                            
% \usepackage{longtable}                                        
% \usepackage{calc}                                             
% \usepackage{multirow}                                         
% \usepackage{hhline}                                           
% \usepackage{ifthen}
% \usepackage{caption} 
% \captionsetup[table]{skip=3pt}  
% \providecommand{\pr}[1]{\ensuremath{\Pr\left(#1\right)}}
% \providecommand{\cbrak}[1]{\ensuremath{\left\{#1\right\}}}
% %\renewcommand{\thefigure}{\arabic{table}}
% \renewcommand{\thetable}{\arabic{table}}      

\setbeamertemplate{caption}[numbered]{}

\usepackage{tfrupee}
\usepackage{amsmath}
\usepackage{amssymb}
\usepackage{gensymb}
\usepackage{graphicx}
\usepackage{txfonts}

\def\inputGnumericTable{}

\usepackage[latin1]{inputenc}                                 
\usepackage{color}                                            
\usepackage{array}                                            
\usepackage{longtable}                                        
\usepackage{calc}                                             
\usepackage{multirow}                                         
\usepackage{hhline}                                           
\usepackage{ifthen}
\usepackage{caption} 
\providecommand{\pr}[1]{\ensuremath{\Pr\left(#1\right)}}
\providecommand{\cbrak}[1]{\ensuremath{\left\{#1\right\}}}
\renewcommand{\thefigure}{\arabic{table}}
\renewcommand{\thetable}{\arabic{table}}   
\providecommand{\brak}[1]{\ensuremath{\left(#1\right)}}

% Theme choice:
\usetheme{CambridgeUS}

% Title page details: 
\title{Assignment 12} 
\author{Rahul Ramachandran}
\date{\today}
% \logo{\large \LaTeX{}}


\begin{document}

% Title page frame
\begin{frame}
    \titlepage 
\end{frame}

% Remove logo from the next slides
\logo{}


% Outline frame
\begin{frame}{Outline}
    \tableofcontents
\end{frame}



\section{Problem Statement}
\begin{frame}{Problem Statement}
    \begin{block}{Papoulis 6.44}  $X$ and $Y$ are independent, identically distributed binomial random variables with parameters $n$ and $p$. Show that $Z = X + Y$ is also a binomial random variable. Find its parameters. \end{block}
\end{frame}

\section{Definitions}
\begin{frame}{$Z-$transform}
 Let the $Z-$transform of the discrete random variable $X$ be defined as
 \begin{align}
     \mathcal{M}_X(z) &= E(z^{-X}) \\
                      &= \sum_{k=-\infty} ^\infty {z^{-k} p_X(k)}
 \end{align}
 
    When $X$ is a binomial random variable with parameters $n$ and $p$:
    
 \begin{align}
     \mathcal{M}_X(z) = \sum_{k=0} ^n {z^{-k} \binom{n}{k} p^k q^{n-k}} 
 \end{align}                   
 
    

\end{frame}


\section{Solution}
\begin{frame}{Solution}
 Events $X$ and $Y$ are given to be independent. Then:
 \begin{align}
      \mathcal{M}_Z(z) &= \mathcal{M}_{X+Y}(z) \\
      &= E(z^{-(X+Y)}) \\
      &=  E(z^{-X}) E(z^{-Y}) \\
      \label{eq7}
      &= \brak{\sum_{k=0} ^n {z^{-k} \binom{n}{k} p^k q^{n-k}}} \brak{ \sum_{l=0} ^n {z^{-l} \binom{n}{l} p^l q^{n-l}} }
 \end{align}
\end{frame}

\begin{frame}{Solution}
    Converting the summation in Equation \ref{eq7} to binomials, we get:
    \begin{align}
      \mathcal{M}_Z(z) &= (q+pz^{-1})^n (q+pz^{-1})^n \\
      &= (q+pz^{-1})^{2n}
 \end{align}
 
 Expanding this, we get:
 
 \begin{align}
        \label{eq10}
      \mathcal{M}_Z(z) &= \sum_{k=0} ^{2n} {z^{-k} \binom{2n}{k} p^k q^{2n-k}}
 \end{align}
 
 
\end{frame}


\begin{frame}{Solution}
Since there is a one-to-one correspondence between the $Z-$transform and the random variable's pmf, it follows from Equation \ref{eq10} that:

\begin{align}
        p_Z(k) = \binom{2n}{k} p^k q^{2n-k} ,~ 0 \le k \le 2n
 \end{align}

Therefore, it follows that $Z$ is a binomial random variable.

\end{frame}



\end{document}