%%%%%%%%%%%%%%%%%%%%%%%%%%%%%%%%%%%%%%%%%%%%%%%%%%%%%%%%%%%%%%%
%
% Welcome to Overleaf --- just edit your LaTeX on the left,
% and we'll compile it for you on the right. If you open the
% 'Share' menu, you can invite other users to edit at the same
% time. See www.overleaf.com/learn for more info. Enjoy!
%
%%%%%%%%%%%%%%%%%%%%%%%%%%%%%%%%%%%%%%%%%%%%%%%%%%%%%%%%%%%%%%%

% Inbuilt themes in beamer
\documentclass{beamer}

%packages:
% \usepackage{tfrupee}
% \usepackage{amsmath}
% \usepackage{amssymb}
% \usepackage{gensymb}
% \usepackage{txfonts}

% \def\inputGnumericTable{}

% \usepackage[latin1]{inputenc}                                 
% \usepackage{color}                                            
% \usepackage{array}                                            
% \usepackage{longtable}                                        
% \usepackage{calc}                                             
% \usepackage{multirow}                                         
% \usepackage{hhline}                                           
% \usepackage{ifthen}
% \usepackage{caption} 
% \captionsetup[table]{skip=3pt}  
% \providecommand{\pr}[1]{\ensuremath{\Pr\left(#1\right)}}
% \providecommand{\cbrak}[1]{\ensuremath{\left\{#1\right\}}}
% %\renewcommand{\thefigure}{\arabic{table}}
% \renewcommand{\thetable}{\arabic{table}}      

\setbeamertemplate{caption}[numbered]{}

\usepackage{enumitem}
\usepackage{tfrupee}
\usepackage{amsmath}
\usepackage{amssymb}
\usepackage{gensymb}
\usepackage{graphicx}
\usepackage{txfonts}

\def\inputGnumericTable{}

\usepackage[latin1]{inputenc}                                 
\usepackage{color}                                            
\usepackage{array}                                            
\usepackage{longtable}                                        
\usepackage{calc}                                             
\usepackage{multirow}                                         
\usepackage{hhline}                                           
\usepackage{ifthen}
\usepackage{caption} 
\captionsetup[table]{skip=3pt}  
\providecommand{\pr}[1]{\ensuremath{\Pr\left(#1\right)}}
\providecommand{\cbrak}[1]{\ensuremath{\left\{#1\right\}}}
\renewcommand{\thefigure}{\arabic{table}}
\renewcommand{\thetable}{\arabic{table}}   
\providecommand{\brak}[1]{\ensuremath{\left(#1\right)}}

% Theme choice:
\usetheme{CambridgeUS}

% Title page details: 
\title{Assignment 9} 
\author{Rahul Ramachandran}
\date{\today}
% \logo{\large \LaTeX{}}


\begin{document}

% Title page frame
\begin{frame}
    \titlepage 
\end{frame}

% Remove logo from the next slides
\logo{}


% Outline frame
\begin{frame}{Outline}
    \tableofcontents
\end{frame}



\section{Problem Statement}
\begin{frame}{Problem Statement}
    \begin{block}{13.5 Q5 [NCERT 12] } The probability that a bulb produced by a factory will fuse after 150 days of use is 0.05. Find the probability that out of 5 such bulbs:
    \begin{enumerate}[label=(\roman*)]
	\item none
	\item not more than one
	\item more than one
	\item at least one
	\end{enumerate}
	will fuse after 150 days of use.
    \end{block}
\end{frame}

\section{Definitions}
\begin{frame}{Random Variable Definition}
Since there are $5$ bulbs, it is appropriate to define a Binomial Random Variable $X$ as under:


\begin{table}[ht!]
    \centering
    \begin{tabular}{|c|c|c|}
\hline
\textbf{Parameter} & \textbf{Formula} & \textbf{Value} \\
\hline
number of shares & \(x\) & ? \\
\hline
value of shares & \(r\) & 50 \\
\hline
cost of shares & \(c\) & 60 \\
\hline
rate of dividend & \(d\) & 10\% \\
\hline
annual income &  \( x \cdot r \cdot d \) & 450 \\
\hline
investment & \(x \cdot c \) & ? \\
\hline
yield percent &  \( \displaystyle{ \frac{\text{income}}{\text{investment}}} \cdot 100 \) & ? \\
\hline

\end{tabular}
    \caption{Random Variable $X$}
	\label{table:table1}
\end{table}

\end{frame}

\begin{frame}{Probability Mass Function}

The probability that a bulb fuses equals $p=0.05$.

Therefore, the probability that $X$ maps to $i$ is given by:
\begin{block}{}
       \begin{align}
                \label{eq1}
           \pr{X=i} = \binom{5}{i} (1-p)^{5-i} p^i ,~ 0 \le i \le 5 
       \end{align}
\end{block}

The values for $i$ can be substituted in the above formula, and the graph of the PMF can be obtained.
\end{frame}


\begin{frame}{Cumulative Distribution Function}
The cumulative probability $ \pr{X \leq i}$ can be defined as under:

\begin{block}{}
\begin{align}
          \label{eq2}
       \pr{X \leq i} = \sum_{k=0}^{i} \binom{5}{k} (1-p)^{5-k} p^k ,~ 0 \le i \le 5
\end{align}
\end{block}

The values of $i$ can be substituted in the above equation, and the obtained values can be used to plot the CDF graph.

\end{frame}


\section{Solution}
\begin{frame}{Solution}
\begin{enumerate}[label=(\roman*)]
  \setcounter{enumi}{0}
\item The probability to be found corresponds to $\pr{X=0}$. Substituting $i=0$ in Equation \ref{eq1}, we get
\begin{align}
    \pr{X=0} &= \binom{5}{0} \times {(1-p)}^{5-0} \times p^0 \\
             &= 1 \times \brak{1-0.05}^{5} \times \brak{0.05}^0 \\
             \label{eq5}
             &\approx 0.774
\end{align}
\end{enumerate}
\end{frame}

\begin{frame}{Solution}
\begin{enumerate}[label=(\roman*)]
  \setcounter{enumi}{1}
\item The probability to be found corresponds to $\pr{X=0}+\pr{X=1}$. Simple addition will give the probability as the events are mutually exclusive. Substituting $i=1$ in Equation \ref{eq1}, we get
\begin{align}
    \pr{X=1} &= \binom{5}{1} \times {(1-p)}^{5-1} \times p^1 \\
             &= 5 \times \brak{1-0.05}^{4} \times \brak{0.05}^1 \\
             &\approx 0.204
\end{align}
    Using the value of \pr{X=0} obtained from equation \ref{eq5}:
    \begin{align}
    \pr{X=0} + \pr{X=1} &\approx 0.774+0.204 \\
             &= 0.978
\end{align}
\end{enumerate}
\end{frame}


\begin{frame}{Solution}
\begin{enumerate}[label=(\roman*)]
  \setcounter{enumi}{2}
\item The probability to be found corresponds to $\pr{X>1}$. This is equivalent to $1-\pr{X<=1}$ (Since $X>1$ and $X<=1$ are mutually exclusive, and the sum of the probabilities is 1). Substituting $i=1$ in Equation \ref{eq2}, we get
\begin{align}
    \pr{X<=1} &= \sum_{k=0}^{1} \binom{5}{k} (1-0.05)^{5-k} 0.05^k \\
             &= 1 \times \brak{1-0.05}^{5} \times \brak{0.05}^0 +5 \times \brak{1-0.05}^{4} \times \brak{0.05}^1 \\
             &\approx 0.774+0.204 \\
             &= 0.978
\end{align}
Therefore,
\begin{align}
    \pr{X>1} &\approx 1-0.978 \\
             &= 0.022
\end{align}

\end{enumerate}
\end{frame}


\begin{frame}{Solution}
\begin{enumerate}[label=(\roman*)]
  \setcounter{enumi}{3}
\item The probability to be found corresponds to $\pr{X>=1}$. This is equivalent to $1-\pr{X<1} = 1-\pr{X=0}$ (Since $X<1$ and $X>=1$ are mutually exclusive).

Substituting the value found in \ref{eq5}:
\begin{align}
    1-\pr{X=0} &\approx 1-0.774 \\
             &= 0.226
\end{align}
\end{enumerate}
\end{frame}

\section{Graphs}
\begin{frame}{PMF Graph}
The PMF graph is:
    \begin{figure}[!ht]
		\centering
		\includegraphics[width=\textwidth,height=5.5cm,keepaspectratio]{figures/PMF.png}
		\caption{Probability Mass Function}
		\label{fig1}
	\end{figure}
\end{frame}

\begin{frame}{CDF Graph}
The CDF graph is:
    \begin{figure}[!ht]
		\centering
		\includegraphics[width=\textwidth,height=5.5cm,keepaspectratio]{figures/CDF.png}
		\caption{Cumulative Distribution Function}
		\label{fig2}
	\end{figure}
\end{frame}


\end{document}