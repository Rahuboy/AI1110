\documentclass[journal,12pt,twocolumn]{IEEEtran}

\usepackage[margin = 1.2in]{geometry}
\usepackage{amsmath,amssymb}
\usepackage{tfrupee}  

\begin{document}

\title{Assignment 1}
\author{Rahul Ramachandran (cs21btech11049)}
\maketitle

\begin{flushleft}
\textbf{5(b) [ICSE 10 2017] :} How much should a man invest in \rupee~50 shares selling at \rupee~60 to obtain an income of \rupee~450, if the rate of dividend declared is 10\%? Also find his yield percent, to the nearest whole number.
\vspace{3mm}

\textbf{Solution:} Let the number of shares the man buys be \(x\). If the shares are worth \(r\) and the dividend is \(d\), then the income is given by \( x \cdot r \cdot d \).

The various parameters involved in this question are listed in the table below:
\begin{table}[ht!]
    \begin{tabular}{|c|c|c|}
\hline
\textbf{Parameter} & \textbf{Formula} & \textbf{Value} \\
\hline
number of shares & \(x\) & ? \\
\hline
value of shares & \(r\) & 50 \\
\hline
cost of shares & \(c\) & 60 \\
\hline
rate of dividend & \(d\) & 10\% \\
\hline
annual income &  \( x \cdot r \cdot d \) & 450 \\
\hline
investment & \(x \cdot c \) & ? \\
\hline
yield percent &  \( \displaystyle{ \frac{\text{income}}{\text{investment}}} \cdot 100 \) & ? \\
\hline

\end{tabular}
\end{table}


Since the man is investing in \rupee~50 shares with a 10\% rate of dividend, here, \(d = 0.1 \) and \( r = 50 \). Therefore, the income he gets equals \[ x\cdot 50 \cdot 0.1 = 5x\]
For an income of \rupee~450, we must have

 \begin{align*}
     5x &= 450 \\
      x &= 90  
\end{align*}
Therefore, the man purchases 90 shares, and must invest \[ x \cdot c = 90 \cdot 60 = \fbox{\text{\rupee~5400}}\]

\vspace{3mm}
The yield percent is given by \( \displaystyle{\frac{\text{income}}{\text{investment}}} \cdot 100 \) and therefore equals \[ \frac{450}{5400} \cdot 100 = 8.33\% \approx \fbox{8 \%} \] 

\end{flushleft}

\end{document}