%%%%%%%%%%%%%%%%%%%%%%%%%%%%%%%%%%%%%%%%%%%%%%%%%%%%%%%%%%%%%%%
%
% Welcome to Overleaf --- just edit your LaTeX on the left,
% and we'll compile it for you on the right. If you open the
% 'Share' menu, you can invite other users to edit at the same
% time. See www.overleaf.com/learn for more info. Enjoy!
%
%%%%%%%%%%%%%%%%%%%%%%%%%%%%%%%%%%%%%%%%%%%%%%%%%%%%%%%%%%%%%%%

% Inbuilt themes in beamer
\documentclass{beamer}

%packages:
% \usepackage{tfrupee}
% \usepackage{amsmath}
% \usepackage{amssymb}
% \usepackage{gensymb}
% \usepackage{txfonts}

% \def\inputGnumericTable{}

% \usepackage[latin1]{inputenc}                                 
% \usepackage{color}                                            
% \usepackage{array}                                            
% \usepackage{longtable}                                        
% \usepackage{calc}                                             
% \usepackage{multirow}                                         
% \usepackage{hhline}                                           
% \usepackage{ifthen}
% \usepackage{caption} 
% \captionsetup[table]{skip=3pt}  
% \providecommand{\pr}[1]{\ensuremath{\Pr\left(#1\right)}}
% \providecommand{\cbrak}[1]{\ensuremath{\left\{#1\right\}}}
% %\renewcommand{\thefigure}{\arabic{table}}
% \renewcommand{\thetable}{\arabic{table}}      

\setbeamertemplate{caption}[numbered]{}

\usepackage{tfrupee}
\usepackage{amsmath}
\usepackage{amssymb}
\usepackage{gensymb}
\usepackage{graphicx}
\usepackage{txfonts}

\def\inputGnumericTable{}

\usepackage[latin1]{inputenc}                                 
\usepackage{color}                                            
\usepackage{array}                                            
\usepackage{longtable}                                        
\usepackage{calc}                                             
\usepackage{multirow}                                         
\usepackage{hhline}                                           
\usepackage{ifthen}
\usepackage{caption} 
\providecommand{\pr}[1]{\ensuremath{\Pr\left(#1\right)}}
\providecommand{\cbrak}[1]{\ensuremath{\left\{#1\right\}}}
\renewcommand{\thefigure}{\arabic{table}}
\renewcommand{\thetable}{\arabic{table}}   
\providecommand{\brak}[1]{\ensuremath{\left(#1\right)}}

% Theme choice:
\usetheme{CambridgeUS}

% Title page details: 
\title{Assignment 11} 
\author{Rahul Ramachandran}
\date{\today}
% \logo{\large \LaTeX{}}


\begin{document}

% Title page frame
\begin{frame}
    \titlepage 
\end{frame}

% Remove logo from the next slides
\logo{}


% Outline frame
\begin{frame}{Outline}
    \tableofcontents
\end{frame}



\section{Problem Statement}
\begin{frame}{Problem Statement}
    \begin{block}{Papoulis 5.50 } A biased coin is tossed and the first outcome is noted. The tossing is continued until the outcome is the complement of the first outcome, thus completing the first run. Let $X$ denote the length of the first run. Find the p.m.f of $X$, and show that   $$ E\{ X \} = \frac{p}{q} + \frac{q}{p}  $$   \end{block}
\end{frame}

\section{Definitions}
\begin{frame}{Definitions}
\begin{itemize}
    \item As specified in the question, we let the random variable $X$ map to the set $\{ 1,2, \ldots \}$ based on the length of the run. We take the length to equal the number of consecutive heads or tails.
    \item We let $H$ represent the event of "heads" and $T$ represent the event of "tails". A string of $H$s and $T$s represents consecutive heads and tails (for example, $HT$ represents heads and then tails).
    \item Let $\pr{H}=p = 1-q$ 
\end{itemize}

\end{frame}


\section{Solution}
\begin{frame}{Solution}
 Consider the case when $X$ maps to $1$. This represents two distinct events, $HT$ and $TH$. Since these events are mutually exclusive, we have:
 \begin{align}
     \pr{X=1} &= \pr{HT+TH} \\
              &= \pr{HT} + \pr{TH} \\
              &= \pr{H} \pr{T} + \pr{H} \pr{T} \\
              &= 2pq
 \end{align}
\end{frame}

\begin{frame}{Solution}
    We can thus similarly find the probability mass function as under
    \begin{align}
     p_X(k)&=\pr{X=k} \\
           &= \pr{TT \ldots TH + HH \ldots HT} \\
           &= \pr{TT \ldots TH} + \pr{HH \ldots HT} \\
           &= q^kp + p^kq
 \end{align}
\end{frame}


\begin{frame}{Solution}
From this, the expectation value $E(X)$ is given by:
\begin{align}
    E(X) &= \sum_{k=1} ^ \infty {k \times p_X(k)} \\
         &= \sum_{k=1} ^ \infty {k \times (q^kp + p^kq)} \\
         \label{eq11}
         &= pq \times \brak{\sum_{k=1} ^ \infty {kp^{k-1}} + \sum_{k=1} ^ \infty {kq^{k-1}} }
\end{align}

\end{frame}

\begin{frame}{Solution}
Equation \ref{eq11} can be manipulated as under:
\begin{align}
   pq \times \brak{\sum_{k=1} ^ \infty {kp^{k-1}} + \sum_{k=1} ^ \infty {kq^{k-1}} } &=  pq \times \brak{ \frac{d}{dp} \sum_{k=1} ^ \infty {p^{k}} + \frac{d}{dq} \sum_{k=1} ^ \infty {q^{k}}} \\
     &= pq \times \brak{ \frac{d}{dp} \brak{\frac{p}{1-p}} + \frac{d}{dq} \brak{\frac{q}{1-q}}} \\
     &= pq \times \brak{\frac{1}{(1-q)^2}+\frac{1}{(1-p)^2}} \\
     &= pq \times \brak{\frac{1}{p^2}+\frac{1}{q^2}} \\
     &= \frac{p}{q} + \frac{q}{p}
\end{align}

\end{frame}





\end{document}T