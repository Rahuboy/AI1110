% Inbuilt themes in beamer
\documentclass{beamer}

%packages:
% \usepackage{tfrupee}
% \usepackage{amsmath}
% \usepackage{amssymb}
% \usepackage{gensymb}
% \usepackage{txfonts}

% \def\inputGnumericTable{}

% \usepackage[latin1]{inputenc}                                 
% \usepackage{color}                                            
% \usepackage{array}                                            
% \usepackage{longtable}                                        
% \usepackage{calc}                                             
% \usepackage{multirow}                                         
% \usepackage{hhline}                                           
% \usepackage{ifthen}
% \usepackage{caption} 
% \captionsetup[table]{skip=3pt}  
% \providecommand{\pr}[1]{\ensuremath{\Pr\left(#1\right)}}
% \providecommand{\cbrak}[1]{\ensuremath{\left\{#1\right\}}}
% %\renewcommand{\thefigure}{\arabic{table}}
% \renewcommand{\thetable}{\arabic{table}}      

\setbeamertemplate{caption}[numbered]{}

\usepackage{tfrupee}
\usepackage{amsmath}
\usepackage{amssymb}
\usepackage{gensymb}
\usepackage{graphicx}
\usepackage{txfonts}

\def\inputGnumericTable{}

\usepackage[latin1]{inputenc}                                 
\usepackage{color}                                            
\usepackage{array}                                            
\usepackage{longtable}                                        
\usepackage{calc}                                             
\usepackage{multirow}                                         
\usepackage{hhline}                                           
\usepackage{ifthen}
\usepackage{caption} 
\providecommand{\pr}[1]{\ensuremath{\Pr\left(#1\right)}}
\providecommand{\cbrak}[1]{\ensuremath{\left\{#1\right\}}}
\renewcommand{\thefigure}{\arabic{table}}
\renewcommand{\thetable}{\arabic{table}}   
\providecommand{\brak}[1]{\ensuremath{\left(#1\right)}}

% Theme choice:
\usetheme{CambridgeUS}

% Title page details: 
\title{Random Numbers} 
\author{Rahul Ramachandran}
\date{\today}
% \logo{\large \LaTeX{}}


\begin{document}

% Title page frame
\begin{frame}
    \titlepage 
\end{frame}

% Remove logo from the next slides
\logo{}


% Outline frame
\begin{frame}{Outline}
    \tableofcontents
\end{frame}



\section{Problem Statement}
\begin{frame}{Problem Statement}
    \begin{block}{(1.5)} Verify the results for the mean and variance of a uniform distribution theoretically given that $E[U^k] = \int_{-\infty}^{\infty}x^k dF_U(x)$
    \end{block}
\end{frame}

\section{Solution}
\begin{frame}{Solution}

Since 
\begin{align}
    dF_U(x) = p_U(x) dx
\end{align}
we have:
\begin{align}
    \label{eq2}
    E[U^k] = \int_{-\infty}^{\infty}x^k p_U(x) dx
\end{align}
Also,
\begin{align}
    \label{eq3}
     p_U(x) = 
    \begin{cases}
        0, & x \in (-\infty,0) \\
        1, & x \in (0,1) \\
        0, & x \in (1, \infty)
    \end{cases}
\end{align}
\end{frame}

\begin{frame}{Solution}
Therefore, from Equations \ref{eq2} and \ref{eq3}, we have:

\begin{align}
    E[U^2] &=  \int_{-\infty}^{\infty}x^2 p_U(x) dx \\
    &= \int_0 ^1 x^2 dx \\
    &= \frac{1}{3}
\end{align}

Similarly, 
\begin{align}
    E[U^2] &=  \int_{-\infty}^{\infty}x p_U(x) dx \\
    &= \int_0 ^1 x dx \\
    &= \frac{1}{2}
\end{align}
\end{frame}

\begin{frame}{Solution}
Therefore, the mean is $\frac{1}{2}$, and the variance equals:
\begin{align}
    E[U^2] - E[U]^2 &= \frac{1}{3} - \brak{\frac{1}{2}}^2 \\
    &= \frac{1}{12}
\end{align}


\end{frame}




\end{document}

\end{document}
