%%%%%%%%%%%%%%%%%%%%%%%%%%%%%%%%%%%%%%%%%%%%%%%%%%%%%%%%%%%%%%%
%
% Welcome to Overleaf --- just edit your LaTeX on the left,
% and we'll compile it for you on the right. If you open the
% 'Share' menu, you can invite other users to edit at the same
% time. See www.overleaf.com/learn for more info. Enjoy!
%
%%%%%%%%%%%%%%%%%%%%%%%%%%%%%%%%%%%%%%%%%%%%%%%%%%%%%%%%%%%%%%%

% Inbuilt themes in beamer
\documentclass{beamer}

%packages:
% \usepackage{tfrupee}
% \usepackage{amsmath}
% \usepackage{amssymb}
% \usepackage{gensymb}
% \usepackage{txfonts}

% \def\inputGnumericTable{}

% \usepackage[latin1]{inputenc}                                 
% \usepackage{color}                                            
% \usepackage{array}                                            
% \usepackage{longtable}                                        
% \usepackage{calc}                                             
% \usepackage{multirow}                                         
% \usepackage{hhline}                                           
% \usepackage{ifthen}
% \usepackage{caption} 
% \captionsetup[table]{skip=3pt}  
% \providecommand{\pr}[1]{\ensuremath{\Pr\left(#1\right)}}
% \providecommand{\cbrak}[1]{\ensuremath{\left\{#1\right\}}}
% %\renewcommand{\thefigure}{\arabic{table}}
% \renewcommand{\thetable}{\arabic{table}}      

\setbeamertemplate{caption}[numbered]{}

\usepackage{tfrupee}
\usepackage{amsmath}
\usepackage{amssymb}
\usepackage{gensymb}
\usepackage{graphicx}
\usepackage{txfonts}

\def\inputGnumericTable{}

\usepackage[latin1]{inputenc}                                 
\usepackage{color}                                            
\usepackage{array}                                            
\usepackage{longtable}                                        
\usepackage{calc}                                             
\usepackage{multirow}                                         
\usepackage{hhline}                                           
\usepackage{ifthen}
\usepackage{caption} 
\providecommand{\pr}[1]{\ensuremath{\Pr\left(#1\right)}}
\providecommand{\cbrak}[1]{\ensuremath{\left\{#1\right\}}}
\renewcommand{\thefigure}{\arabic{table}}
\renewcommand{\thetable}{\arabic{table}}   
\providecommand{\brak}[1]{\ensuremath{\left(#1\right)}}

% Theme choice:
\usetheme{CambridgeUS}

% Title page details: 
\title{Assignment 13} 
\author{Rahul Ramachandran}
\date{\today}
% \logo{\large \LaTeX{}}


\begin{document}

% Title page frame
\begin{frame}
    \titlepage 
\end{frame}

% Remove logo from the next slides
\logo{}


% Outline frame
\begin{frame}{Outline}
    \tableofcontents
\end{frame}



\section{Problem Statement}
\begin{frame}{Problem Statement}
    \begin{block}{Papoulis 8.10} Among $4000$ newborns, $2080$ are male. Find the $0.99$ confidence interval of the probability
$p = P$(male) \end{block}
\end{frame}

\section{Definitions}
\begin{frame}{Definitions}

\begin{itemize}
    \item A \textbf{Confidence Interval} is an estimate for an unknown parameter. It is governed by a number $\gamma = 1 - \delta$, which determines the accuracy of the estimation method. $\gamma$ is called the \textbf{confidence coefficient}.
    \item A \textbf{$z$-score} is the number of standard deviations a data point is away from the mean.
    \item If $\hat{p}$ is a sample mean, the distribution of $\hat{p}$ is called the \textbf{sampling distribution of the sample means}.
    \item To find the confidence interval, we find the points on the $x-$axis between which the area under the curve equals $\gamma A$
\end{itemize}


    
\end{frame}

\section{Discussion}
\begin{frame}{Binomial Distribution Mean}
Let the random variable $X$ be the sum of $n$ Bernoulli random variables, i.e.
\begin{align}
     X = X_1 + X_2 + \ldots +X_n
 \end{align}
 For simplicity, assume $E(X_i) = p$. Then, 
 \begin{align}
     E(X) &= E(X_1 + X_2 + \ldots +X_n) \\
     &= E(X_1) + E(X_2) + \ldots + E(X_n) \\
     &= p + p + \ldots + p \\
     \label{eq5}
     &= np
 \end{align}


\end{frame}


\begin{frame}{Binomial Distribution Variance}
The variance is given by:

 \begin{align}
     E((X-np)^2) &= E(X^2-2Xnp+n^2p^2) \\
     &= E(X^2) -2npE(X) + E(n^2p^2) \\
     \label{eq8}
     &= E(X^2) - n^2p^2 
 \end{align}
 Now,
 \begin{align}
     E(X^2) &= \sum _{k=0} ^n k^2 \times \binom{n}{k} p^k q^{n-k}\\
     &= \sum _{k=1} ^n npk \times \binom{n-1}{k-1} p^{k-1} q^{n-k} 
 \end{align}


\end{frame}

\begin{frame}{Binomial Distribution Variance}
Since $k = (k-1) + 1$, we have:

 \begin{align}
     \sum _{k=1} ^n npk \times \binom{n-1}{k-1} p^{k-1} q^{n-k}  &=  n(n-1)p^2  \times \sum_{k=2} ^n\binom{n-2}{k-2} p^{k-2}q^{n-k}  \nonumber \\&+ np \times \sum_{k=1} ^n \binom{n-1}{k-1} p^{k-1} q^{n-k}\\
     &= n(n-1)p^2  (p+q)^{n-2} + np  (p+q)^{n-1} \\
     &= n(n-1)p^2 + np
 \end{align}
Substituting this in Equation \ref{eq8}, we have:

\begin{align}
    \sigma^2 &= (n(n-1)p^2 + np) -n^2p^2 \\
    &= np - np^2 
\end{align}

\end{frame}

\begin{frame}{Binomial Distribution}
Therefore:
    \begin{align}
        \sigma^2 &= np-np^2 \\
        &= np(1-p) \\
        \label{eq18}
        &= npq
    \end{align}
Let the random variable $Y = \frac{X}{n}$. Then, $Y$ is a binomial random variable that maps to $\frac{k}{n}$ when $X$ maps to $k$. Therefore, 
\begin{align}
        \mu_Y &= \frac{\mu_X}{n} \\
        &= p \\
        \sigma^2_Y &= \frac{\sigma^2_X}{n^2} \\
           &= \frac{pq}{n}
    \end{align}
\end{frame}


\begin{frame}{De Moivre-Laplace Theorem}
    The \textbf{De Moivre-Laplace Theorem} states that for large $n$ (and a significant value of $np$), the Normal Distribution can be used to approximate a Binomial Distribution.
    
    Let the point $k$ in a Binomial distribution equal $\mu + c\sigma$, where $c$ is an arbitrary constant. Therefore, from Equation \ref{eq5} and Equation \ref{eq18}:
    
    \begin{align}
        k = np + c \sqrt{npq}
    \end{align}
    
    The differential equation for a Normal Distribution is:
    
    \begin{align}
        \label{eq24}
        \frac{f'(x)}{f(x)} \brak{-\frac{\sigma^2}{x-\mu}} = 1
    \end{align}
    
    
\end{frame}

\begin{frame}{De Moivre-Laplace Theorem}
    If we show that the pdf of the Binomial Distribution satisfies Equation \ref{eq24}, we have proved the theorem.
    Since the Binomial Distribution is discrete, we consider the discrete analogue of the derivative, $p(k+1)-p(k)$.
    
    \begin{align}
          \frac{f'(x)}{f(x)} \brak{-\frac{\sigma^2}{x-\mu}} &=   \frac{p(k+1)-p(k)}{p(k)} \brak{\frac{\sqrt{npq}}{-c}} \\
          &= \frac{\binom{n}{k+1}p^{k+1}q^{n-(k+1)} + \binom{n}{k}p^{k}q^{n-k} }{\binom{n}{k}p^{k}q^{n-k}}  \brak{\frac{\sqrt{npq}}{-c}} \\
          &= \brak{\frac{(n-k)p}{(k+1)q}-1}\brak{\frac{\sqrt{npq}}{-c}} \\ 
          &= \frac{np-k-q}{(k+1)q}\brak{\frac{\sqrt{npq}}{-c}}
    \end{align}
    

\end{frame}

\begin{frame}{De Moivre-Laplace Theorem}
    Substituting for $k$, we get:
    
    \begin{align}
        \frac{f'(x)}{f(x)} \brak{-\frac{\sigma^2}{x-\mu}} &= \frac{-c\sqrt{npq}-q}{npq+cq\sqrt{npq}+q}\brak{\frac{\sqrt{npq}}{-c}}
    \end{align}
    As $n \rightarrow \infty$
    \begin{align}
        \frac{-c\sqrt{npq}-q}{npq+cq\sqrt{npq}+q}\brak{\frac{\sqrt{npq}}{-c}} &\rightarrow 1
    \end{align}
    
    Therefore, the De Moivre-Laplace Theorem holds.
    

\end{frame}

\begin{frame}{Confidence Interval}
    Consider a Normal Distribution of sample means symmetric about $x=p$. Let the number $z$ be chosen such that the area between $x=-\infty$ and $x=\mu + z\sigma = A(1-\frac{\delta}{2})$, where $A$ is the area under the whole curve.
    
    Then: 
    \begin{align}
        Ar(\mu - z\sigma \leq x \leq \mu + z\sigma) &= A(1-\delta) \\
        \implies \pr{p - z\sigma \leq \hat{p} \leq p + z\sigma } &= 1-\delta
    \end{align}
Analogously, for a Normal Distribution symmetric about $\hat{p}$, with variance $\sigma^2$,

\begin{align}
       \pr{\hat{p} - z\sigma \leq p \leq \hat{p} + z\sigma } &= 1-\delta
    \end{align}

\end{frame}

\begin{frame}{Confidence Interval}
    The interval $[\hat{p} - z\sigma,   \hat{p} + z\sigma ]$ is called the  $\gamma$ confidence interval of $\hat{p}$. Often, we aren't aware of the exact value of $p$, so we approximate it as under:
    
    \begin{align}
        \sigma &=  \sqrt{\frac{pq}{n}} \\
        &\approx \sqrt{\frac{\hat{p}\hat{q}}{n}}
    \end{align}

Since all the values are now known, we can find an estimate for the confidence interval of $\hat{p}$

\end{frame}


\section{Solution}
\begin{frame}{Solution}
Since there are $4000$ babies and $2080$ girls, for the sample, we have:
\begin{align}
    \hat{p} &= 2080/4000 \\
            &= 0.52
 \end{align}
 Also,
     \begin{align}
    \sigma_{\hat{p}} &= \sqrt{\frac{0.52(1-0.52)}{4000}} \\
    &= 0.0079
 \end{align}


\end{frame}

\begin{frame}{Solution}
    
    Since we are interested in the $0.99$ confidence interval, we have
    
     \begin{align}
     \gamma &= 0.99 \\ 
     \implies \delta &= 1 - \gamma \\
     &= 0.01 
 \end{align}
 
 Therefore, we have to find $z$ corresponding to $\delta = 0.01$, which from the $z$-score table equals $2.58$
    
\end{frame}


\begin{frame}{Solution}
    Therefore, it follows that
    \begin{align}
        p_u &= \mu + z\sigma \\
        &= 0.52 + 2.58 \times 0.0079 \\
        &= 0.54
    \end{align}
    where $p_u$ is the upper limit of the interval. Similarly, 
    \begin{align}
        p_l &= \mu - z\sigma \\
        &= 0.52 - 2.58 \times 0.0079 \\
        &= 0.49
    \end{align}
    Therefore, the $0.99$ confidence interval for $p$ is $[0.49,0.54]$
\end{frame}

\end{document}
