%%%%%%%%%%%%%%%%%%%%%%%%%%%%%%%%%%%%%%%%%%%%%%%%%%%%%%%%%%%%%%%
%
% Welcome to Overleaf --- just edit your LaTeX on the left,
% and we'll compile it for you on the right. If you open the
% 'Share' menu, you can invite other users to edit at the same
% time. See www.overleaf.com/learn for more info. Enjoy!
%
%%%%%%%%%%%%%%%%%%%%%%%%%%%%%%%%%%%%%%%%%%%%%%%%%%%%%%%%%%%%%%%

% Inbuilt themes in beamer
\documentclass{beamer}

%packages:
% \usepackage{tfrupee}
% \usepackage{amsmath}
% \usepackage{amssymb}
% \usepackage{gensymb}
% \usepackage{txfonts}

% \def\inputGnumericTable{}

% \usepackage[latin1]{inputenc}                                 
% \usepackage{color}                                            
% \usepackage{array}                                            
% \usepackage{longtable}                                        
% \usepackage{calc}                                             
% \usepackage{multirow}                                         
% \usepackage{hhline}                                           
% \usepackage{ifthen}
% \usepackage{caption} 
% \captionsetup[table]{skip=3pt}  
% \providecommand{\pr}[1]{\ensuremath{\Pr\left(#1\right)}}
% \providecommand{\cbrak}[1]{\ensuremath{\left\{#1\right\}}}
% %\renewcommand{\thefigure}{\arabic{table}}
% \renewcommand{\thetable}{\arabic{table}}      

\setbeamertemplate{caption}[numbered]{}

\usepackage{tfrupee}
\usepackage{amsmath}
\usepackage{amssymb}
\usepackage{gensymb}
\usepackage{graphicx}
\usepackage{txfonts}

\def\inputGnumericTable{}

\usepackage[latin1]{inputenc}                                 
\usepackage{color}                                            
\usepackage{array}                                            
\usepackage{longtable}                                        
\usepackage{calc}                                             
\usepackage{multirow}                                         
\usepackage{hhline}                                           
\usepackage{ifthen}
\usepackage{caption} 
\providecommand{\pr}[1]{\ensuremath{\Pr\left(#1\right)}}
\providecommand{\cbrak}[1]{\ensuremath{\left\{#1\right\}}}
\renewcommand{\thefigure}{\arabic{table}}
\renewcommand{\thetable}{\arabic{table}}   
\providecommand{\brak}[1]{\ensuremath{\left(#1\right)}}

% Theme choice:
\usetheme{CambridgeUS}

% Title page details: 
\title{Assignment 14} 
\author{Rahul Ramachandran}
\date{\today}
% \logo{\large \LaTeX{}}


\begin{document}

% Title page frame
\begin{frame}
    \titlepage 
\end{frame}

% Remove logo from the next slides
\logo{}


% Outline frame
\begin{frame}{Outline}
    \tableofcontents
\end{frame}



\section{Problem Statement}
\begin{frame}{Problem Statement}
    \begin{block}{Papoulis 15.4} Show that the probability of extinction of a population given that the zeroth generation has size $m$ is given by $\pi_0^m$, where $\pi_0$ is the extinction probability. Show that the probability that the population grows indefinitely in that case is $1-\pi_0^m$.
    \end{block}
\end{frame}

\section{Definitions}
\begin{frame}{Definitions}

\begin{itemize}
    \item $P_Y(z) = E(z^Y)$ is called the common moment generating function, and is defined as:
    \begin{align}
         P_Y(z) = \sum_{k=0}^\infty p_kz^k
    \end{align}
    where $p_k = \pr{Y=k}$
    \item A \textbf{branching process} is a process consisting of independent reproducing individuals. Each individual lives a single unit of time, and has $Y_{i}$ offspring
    \item $X_n$ maps to the total size of the $n^{\text{th}}$ generation, and equals $Y_1 + Y_2 + \ldots Y_{X_{n-1}}$
    \item The \textbf{extinction probability} $\pi_0$ is the probability that the $k^{\text{th}}$ generation has no offspring, for some finite $k$. It equals $\displaystyle{\lim_{n \rightarrow \infty} P_{X_n}(0)}$
\end{itemize}


    
\end{frame}

\section{Discussion}
\begin{frame}{Common MGF}
We have
\begin{align}
    X_n = \sum_{i=1}^{X_{n-1}} Y_i
\end{align}
where $Y_i$ are i.i.d. Also,
 \begin{align}
     P_{X_n}(z) = \sum_{k=0}^{\infty} \pr{X_n=k} z^k
 \end{align}
 
 This can be rewritten as:
 
 \begin{align}
    \label{eq4}
     P_{X_n}(z) = \sum_{i=0}^{\infty}\sum_{k=0}^{\infty} \pr{X_n=k | X_{n-1} = i} \pr{X_{n-1}=i} z^k
 \end{align}
 
 using the Law of Total Probability.
 
\end{frame}


\begin{frame}{Common MGF}
From \ref{eq4}, we get:

 \begin{align}
     P_{X_n}(z) &= \sum_{i=0}^{\infty} \pr{X_{n-1}=i} \sum_{k=0}^{\infty} \pr{X_n=k | X_{n-1} = i}  z^k \\
 \end{align}
 When $X_n = Y_1 + Y_2 + \ldots Y_i$, observe that: 
 \begin{align}
     E(z^{X_n}) &= E(z^{Y_1 + Y_2 + \ldots Y_i}) \\
     &= [E(z^{Y_1})]^i
 \end{align}
 since $Y_i$ are i.i.d . Therefore, 
 \begin{align}
    P_{X_n}(z) &= \sum_{i=0}^{\infty} \pr{X_{n-1}=i} [P_{Y_1}(z)]^i \\
    &= P_{X_{n-1}}(P_{Y_1}(z))
 \end{align}
 
\end{frame}

\begin{frame}{Common MGF}
Assuming that $X_0 = Y_1$, (assuming the zeroth generation has size 1) we have:
\begin{align}
     P_{X_n}(z) &=  P_{X_{n-1}}(P_{X_0}(z))
\end{align}
Therefore:
\begin{align}
    P_{X_n}(z) &=  P_{X_{n-1}}(P_{X_0}(z)) \\
    &= P_{X_{n-2}}(P_{X_0}(P_{X_0}(z))) \\
    &= P_{X_0}(P_{X_0}(\ldots P_{X_0}(z)) \\
    \label{eq15}
    &= P_{X_0}(P_{X_{n-1}}(z))
\end{align}

\end{frame}

\begin{frame}{Extinction Probability}
From the definition of extinction probability, we have: 
\begin{align}
    \pi_0 &= \lim_{n \rightarrow \infty}  P_{X_n}(0) \\
    &= \lim_{n \rightarrow \infty} P_{X_{n-1}}(P_{X_0}(0))
\end{align}
Using Equation \ref{eq15}:
\begin{align}
     \pi_0 &= \lim_{n \rightarrow \infty}  P_{X_0}(P_{X_{n-1}}(0)) \\
     &= P_{X_0}(\pi_0)
\end{align}
Hence, the extinction probability can be obtained by solving the equation $\pi_0 = P_{X_0}(\pi_0)$
\end{frame}


\section{Solution}
\begin{frame}{Solution}
If the zeroth generation has size $m$, we can split it into $m$ independent and identical generations of size $1$. Therefore, since $\pi_0$ is the extinction probability of a single generation, we have: 
\begin{align}
    p_e &=  \lim_{n \rightarrow \infty} P_{X_0 ^{(1)}}(P_{X_{n-1}^{(1)}}(0)) \times \ldots \times \lim_{n \rightarrow \infty} P_{X_0 ^{(m)}}(P_{X_{n-1}^{(m)}}(0))\\
            &= \pi_0 \times \pi_0 \times \ldots \pi_0 \\
            &= \pi_0^m
 \end{align}
 where $p_e$ is the extinction probability for the specified generation. We then have:
     \begin{align}
    p_s &= 1 - p_e \\
    &= 1 - \pi_0^m 
 \end{align}
 where $p_s$ is the probability that the population will survive and grow indefinitely.


\end{frame}



\end{document}