%%%%%%%%%%%%%%%%%%%%%%%%%%%%%%%%%%%%%%%%%%%%%%%%%%%%%%%%%%%%%%%
%
% Welcome to Overleaf --- just edit your LaTeX on the left,
% and we'll compile it for you on the right. If you open the
% 'Share' menu, you can invite other users to edit at the same
% time. See www.overleaf.com/learn for more info. Enjoy!
%
%%%%%%%%%%%%%%%%%%%%%%%%%%%%%%%%%%%%%%%%%%%%%%%%%%%%%%%%%%%%%%%

% Inbuilt themes in beamer
\documentclass{beamer}

%packages:
% \usepackage{tfrupee}
% \usepackage{amsmath}
% \usepackage{amssymb}
% \usepackage{gensymb}
% \usepackage{txfonts}

% \def\inputGnumericTable{}

% \usepackage[latin1]{inputenc}                                 
% \usepackage{color}                                            
% \usepackage{array}                                            
% \usepackage{longtable}                                        
% \usepackage{calc}                                             
% \usepackage{multirow}                                         
% \usepackage{hhline}                                           
% \usepackage{ifthen}
% \usepackage{caption} 
% \captionsetup[table]{skip=3pt}  
% \providecommand{\pr}[1]{\ensuremath{\Pr\left(#1\right)}}
% \providecommand{\cbrak}[1]{\ensuremath{\left\{#1\right\}}}
% %\renewcommand{\thefigure}{\arabic{table}}
% \renewcommand{\thetable}{\arabic{table}}      

\setbeamertemplate{caption}[numbered]{}

\usepackage{enumitem}
\usepackage{tfrupee}
\usepackage{amsmath}
\usepackage{amssymb}
\usepackage{gensymb}
\usepackage{graphicx}
\usepackage{txfonts}

\def\inputGnumericTable{}

\usepackage[latin1]{inputenc}                                 
\usepackage{color}                                            
\usepackage{array}                                            
\usepackage{longtable}                                        
\usepackage{calc}                                             
\usepackage{multirow}                                         
\usepackage{hhline}                                           
\usepackage{ifthen}
\usepackage{caption} 
\captionsetup[table]{skip=3pt}  
\providecommand{\pr}[1]{\ensuremath{\Pr\left(#1\right)}}
\providecommand{\cbrak}[1]{\ensuremath{\left\{#1\right\}}}
\renewcommand{\thefigure}{\arabic{table}}
\renewcommand{\thetable}{\arabic{table}}   
\providecommand{\brak}[1]{\ensuremath{\left(#1\right)}}

% Theme choice:
\usetheme{CambridgeUS}

% Title page details: 
\title{Assignment 10} 
\author{Rahul Ramachandran}
\date{\today}
% \logo{\large \LaTeX{}}


\begin{document}

% Title page frame
\begin{frame}
    \titlepage 
\end{frame}

% Remove logo from the next slides
\logo{}


% Outline frame
\begin{frame}{Outline}
    \tableofcontents
\end{frame}



\section{Problem Statement}
\begin{frame}{Problem Statement}
    \begin{block}{Papoulis 4.27 } A coin is tossed an infinite number of times. Show that the probability that $k$ heads are observed at the $n^{\text{th}}$ toss but not earlier equals $\binom{n-1}{k-1}p^{k}q^{n-k}$     \end{block}
\end{frame}

\section{Definitions}
\begin{frame}{Events Definition}
Let events $L,M$ and $N$ be defined as under:

\begin{table}[ht!]
    \centering
    \begin{tabular}{|c|c|c|}
\hline
\textbf{Parameter} & \textbf{Formula} & \textbf{Value} \\
\hline
number of shares & \(x\) & ? \\
\hline
value of shares & \(r\) & 50 \\
\hline
cost of shares & \(c\) & 60 \\
\hline
rate of dividend & \(d\) & 10\% \\
\hline
annual income &  \( x \cdot r \cdot d \) & 450 \\
\hline
investment & \(x \cdot c \) & ? \\
\hline
yield percent &  \( \displaystyle{ \frac{\text{income}}{\text{investment}}} \cdot 100 \) & ? \\
\hline

\end{tabular}
	\label{table:table1}
\end{table}

\end{frame}

\begin{frame}{Random Variable Definition}

Define a binomial random variable $X$ with parameters $m$ and $p$.

We then have:
\begin{block}{}
       \begin{align}
                \label{eq1}
           p_X{(i)} = \binom{m}{i} \times p^i \times q^{m-i}  ,~ 0 \le i \le m
       \end{align}
\end{block}

In the present problem, $X$ can be used to find probabilities relevant to event $N$.
\end{frame}

\section{Solution}
\begin{frame}{Solution}
 Observe that we are required to find $\pr{L}$. Also observe that
 \begin{align}
     L=MN
 \end{align}
 Since $M$ and $N$ are independent events, we have:
  \begin{align}
     \pr{L}&=\pr{MN} \\
            \label{eq4}
           &= \pr{M} \pr{N}
 \end{align}
\end{frame}

\begin{frame}{Solution}
    $\pr{N}$ corresponds to $p_X(k-1)$ where the parameters of $X$ are $(n-1,p)$.
    Therefore, using Equation \ref{eq1}:
    \begin{align}
     \pr{N}&=p_X(k-1) \\
           &= \binom{n-1}{k-1}\times p^{k-1} \times q^{n-k}
 \end{align}
\end{frame}


\begin{frame}{Solution}
Now,
\begin{align}
    \pr{M} = p
\end{align}
Therefore, substituting the corresponding values in Equation \ref{eq4}:
\begin{align}
    \pr{L} &= p \times \binom{n-1}{k-1} p^{k-1}  q^{n-k} \\
           &= \binom{n-1}{k-1} p^{k}  q^{n-k} 
\end{align}
\end{frame}

\end{document}T