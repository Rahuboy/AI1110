%%%%%%%%%%%%%%%%%%%%%%%%%%%%%%%%%%%%%%%%%%%%%%%%%%%%%%%%%%%%%%%
%
% Welcome to Overleaf --- just edit your LaTeX on the left,
% and we'll compile it for you on the right. If you open the
% 'Share' menu, you can invite other users to edit at the same
% time. See www.overleaf.com/learn for more info. Enjoy!
%
%%%%%%%%%%%%%%%%%%%%%%%%%%%%%%%%%%%%%%%%%%%%%%%%%%%%%%%%%%%%%%%

% Inbuilt themes in beamer
\documentclass{beamer}

%packages:
% \usepackage{tfrupee}
% \usepackage{amsmath}
% \usepackage{amssymb}
% \usepackage{gensymb}
% \usepackage{txfonts}

% \def\inputGnumericTable{}

% \usepackage[latin1]{inputenc}                                 
% \usepackage{color}                                            
% \usepackage{array}                                            
% \usepackage{longtable}                                        
% \usepackage{calc}                                             
% \usepackage{multirow}                                         
% \usepackage{hhline}                                           
% \usepackage{ifthen}
% \usepackage{caption} 
% \captionsetup[table]{skip=3pt}  
% \providecommand{\pr}[1]{\ensuremath{\Pr\left(#1\right)}}
% \providecommand{\cbrak}[1]{\ensuremath{\left\{#1\right\}}}
% %\renewcommand{\thefigure}{\arabic{table}}
% \renewcommand{\thetable}{\arabic{table}}      

\setbeamertemplate{caption}[numbered]{}

\usepackage{tfrupee}
\usepackage{amsmath}
\usepackage{amssymb}
\usepackage{gensymb}
\usepackage{graphicx}
\usepackage{txfonts}

\def\inputGnumericTable{}

\usepackage[latin1]{inputenc}                                 
\usepackage{color}                                            
\usepackage{array}                                            
\usepackage{longtable}                                        
\usepackage{calc}                                             
\usepackage{multirow}                                         
\usepackage{hhline}                                           
\usepackage{ifthen}
\usepackage{caption} 
\providecommand{\pr}[1]{\ensuremath{\Pr\left(#1\right)}}
\providecommand{\cbrak}[1]{\ensuremath{\left\{#1\right\}}}
\renewcommand{\thefigure}{\arabic{table}}
\renewcommand{\thetable}{\arabic{table}}   
\providecommand{\brak}[1]{\ensuremath{\left(#1\right)}}

% Theme choice:
\usetheme{CambridgeUS}

% Title page details: 
\title{Assignment 13} 
\author{Rahul Ramachandran}
\date{\today}
% \logo{\large \LaTeX{}}


\begin{document}

% Title page frame
\begin{frame}
    \titlepage 
\end{frame}

% Remove logo from the next slides
\logo{}


% Outline frame
\begin{frame}{Outline}
    \tableofcontents
\end{frame}



\section{Problem Statement}
\begin{frame}{Problem Statement}
    \begin{block}{Papoulis 8.10} Among $4000$ newborns, $2080$ are male. Find the $0.99$ confidence interval of the probability
$p = P$(male) \end{block}
\end{frame}

\section{Discussion}
\begin{frame}{Discussion}
 We are given a sample space of babies, and are required to provide an estimate of parameter $p$: the proportion of babies in a population that are male.
 
 Let the random variable $X$ map to 0 when the baby is female, and $1$ otherwise. Let the average value of $X_i$ for the given sample space be $\hat{p}$. We can use $\hat{p} = 1 - \hat{q}$ to estimate an interval that $p$ is likely to lie in.

\end{frame}




\begin{frame}{Solution}
 Since $\hat{p} = \frac{\sum X_i}{n}$, and since $n$ is large, the sampling distribution of sample proportion can be approximated to a normal distribution, by the Central Limit Theorem.
 
 To find the confidence interval, we assume that the mean of this normal distribution is $\hat{p}$, and that the standard deviation is $\sigma_{\hat{p}} = \sqrt{\frac{\hat{p} \hat{q}}{n}}$
\end{frame}

\section{Solution}
\begin{frame}{Solution}
We have
\begin{align}
    \hat{p} &= 2080/4000 \\
            &= 0.52
 \end{align}
 Also,
     \begin{align}
    \sigma_{\hat{p}} &= \sqrt{\frac{0.52(1-0.52)}{4000}} \\
    &= 0.0079
 \end{align}


\end{frame}

\begin{frame}{Solution}
    To find the interval, we use the $z-$score, which tells us the number of standard deviations between the end-points of the confidence interval and the mean.
    Since we are interested in the $0.99$ confidence interval, we have
    
     \begin{align}
     \gamma &= 0.99 \\ 
     \implies \delta &= 1 - \gamma \\
     &= 0.01 
 \end{align}
 
 Therefore, we have to find $z$ corresponding to $\delta = 0.01$, which from the $z$-score table equals $2.58$
    
\end{frame}


\begin{frame}{Solution}
    Therefore, it follows that
    \begin{align}
        p_u &= \mu + z\sigma \\
        &= 0.52 + 2.58 \times 0.0079 \\
        &= 0.54
    \end{align}
    where $p_u$ is the upper limit of the interval. Similarly, 
    \begin{align}
        p_l &= \mu - z\sigma \\
        &= 0.52 - 2.58 \times 0.0079 \\
        &= 0.49
    \end{align}
    Therefore, the $0.99$ confidence interval for $p$ is $[0.49,0.54]$
\end{frame}

\end{document}